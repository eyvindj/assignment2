% Options for packages loaded elsewhere
\PassOptionsToPackage{unicode}{hyperref}
\PassOptionsToPackage{hyphens}{url}
%
\documentclass[
  5pt,
]{article}
\usepackage{amsmath,amssymb}
\usepackage{iftex}
\ifPDFTeX
  \usepackage[T1]{fontenc}
  \usepackage[utf8]{inputenc}
  \usepackage{textcomp} % provide euro and other symbols
\else % if luatex or xetex
  \usepackage{unicode-math} % this also loads fontspec
  \defaultfontfeatures{Scale=MatchLowercase}
  \defaultfontfeatures[\rmfamily]{Ligatures=TeX,Scale=1}
\fi
\usepackage{lmodern}
\ifPDFTeX\else
  % xetex/luatex font selection
\fi
% Use upquote if available, for straight quotes in verbatim environments
\IfFileExists{upquote.sty}{\usepackage{upquote}}{}
\IfFileExists{microtype.sty}{% use microtype if available
  \usepackage[]{microtype}
  \UseMicrotypeSet[protrusion]{basicmath} % disable protrusion for tt fonts
}{}
\makeatletter
\@ifundefined{KOMAClassName}{% if non-KOMA class
  \IfFileExists{parskip.sty}{%
    \usepackage{parskip}
  }{% else
    \setlength{\parindent}{0pt}
    \setlength{\parskip}{6pt plus 2pt minus 1pt}}
}{% if KOMA class
  \KOMAoptions{parskip=half}}
\makeatother
\usepackage{xcolor}
\usepackage[margin=1in]{geometry}
\usepackage{color}
\usepackage{fancyvrb}
\newcommand{\VerbBar}{|}
\newcommand{\VERB}{\Verb[commandchars=\\\{\}]}
\DefineVerbatimEnvironment{Highlighting}{Verbatim}{commandchars=\\\{\}}
% Add ',fontsize=\small' for more characters per line
\usepackage{framed}
\definecolor{shadecolor}{RGB}{248,248,248}
\newenvironment{Shaded}{\begin{snugshade}}{\end{snugshade}}
\newcommand{\AlertTok}[1]{\textcolor[rgb]{0.94,0.16,0.16}{#1}}
\newcommand{\AnnotationTok}[1]{\textcolor[rgb]{0.56,0.35,0.01}{\textbf{\textit{#1}}}}
\newcommand{\AttributeTok}[1]{\textcolor[rgb]{0.13,0.29,0.53}{#1}}
\newcommand{\BaseNTok}[1]{\textcolor[rgb]{0.00,0.00,0.81}{#1}}
\newcommand{\BuiltInTok}[1]{#1}
\newcommand{\CharTok}[1]{\textcolor[rgb]{0.31,0.60,0.02}{#1}}
\newcommand{\CommentTok}[1]{\textcolor[rgb]{0.56,0.35,0.01}{\textit{#1}}}
\newcommand{\CommentVarTok}[1]{\textcolor[rgb]{0.56,0.35,0.01}{\textbf{\textit{#1}}}}
\newcommand{\ConstantTok}[1]{\textcolor[rgb]{0.56,0.35,0.01}{#1}}
\newcommand{\ControlFlowTok}[1]{\textcolor[rgb]{0.13,0.29,0.53}{\textbf{#1}}}
\newcommand{\DataTypeTok}[1]{\textcolor[rgb]{0.13,0.29,0.53}{#1}}
\newcommand{\DecValTok}[1]{\textcolor[rgb]{0.00,0.00,0.81}{#1}}
\newcommand{\DocumentationTok}[1]{\textcolor[rgb]{0.56,0.35,0.01}{\textbf{\textit{#1}}}}
\newcommand{\ErrorTok}[1]{\textcolor[rgb]{0.64,0.00,0.00}{\textbf{#1}}}
\newcommand{\ExtensionTok}[1]{#1}
\newcommand{\FloatTok}[1]{\textcolor[rgb]{0.00,0.00,0.81}{#1}}
\newcommand{\FunctionTok}[1]{\textcolor[rgb]{0.13,0.29,0.53}{\textbf{#1}}}
\newcommand{\ImportTok}[1]{#1}
\newcommand{\InformationTok}[1]{\textcolor[rgb]{0.56,0.35,0.01}{\textbf{\textit{#1}}}}
\newcommand{\KeywordTok}[1]{\textcolor[rgb]{0.13,0.29,0.53}{\textbf{#1}}}
\newcommand{\NormalTok}[1]{#1}
\newcommand{\OperatorTok}[1]{\textcolor[rgb]{0.81,0.36,0.00}{\textbf{#1}}}
\newcommand{\OtherTok}[1]{\textcolor[rgb]{0.56,0.35,0.01}{#1}}
\newcommand{\PreprocessorTok}[1]{\textcolor[rgb]{0.56,0.35,0.01}{\textit{#1}}}
\newcommand{\RegionMarkerTok}[1]{#1}
\newcommand{\SpecialCharTok}[1]{\textcolor[rgb]{0.81,0.36,0.00}{\textbf{#1}}}
\newcommand{\SpecialStringTok}[1]{\textcolor[rgb]{0.31,0.60,0.02}{#1}}
\newcommand{\StringTok}[1]{\textcolor[rgb]{0.31,0.60,0.02}{#1}}
\newcommand{\VariableTok}[1]{\textcolor[rgb]{0.00,0.00,0.00}{#1}}
\newcommand{\VerbatimStringTok}[1]{\textcolor[rgb]{0.31,0.60,0.02}{#1}}
\newcommand{\WarningTok}[1]{\textcolor[rgb]{0.56,0.35,0.01}{\textbf{\textit{#1}}}}
\usepackage{graphicx}
\makeatletter
\def\maxwidth{\ifdim\Gin@nat@width>\linewidth\linewidth\else\Gin@nat@width\fi}
\def\maxheight{\ifdim\Gin@nat@height>\textheight\textheight\else\Gin@nat@height\fi}
\makeatother
% Scale images if necessary, so that they will not overflow the page
% margins by default, and it is still possible to overwrite the defaults
% using explicit options in \includegraphics[width, height, ...]{}
\setkeys{Gin}{width=\maxwidth,height=\maxheight,keepaspectratio}
% Set default figure placement to htbp
\makeatletter
\def\fps@figure{htbp}
\makeatother
\setlength{\emergencystretch}{3em} % prevent overfull lines
\providecommand{\tightlist}{%
  \setlength{\itemsep}{0pt}\setlength{\parskip}{0pt}}
\setcounter{secnumdepth}{-\maxdimen} % remove section numbering
\usepackage{booktabs}
\usepackage{longtable}
\usepackage{array}
\usepackage{multirow}
\usepackage{wrapfig}
\usepackage{float}
\usepackage{colortbl}
\usepackage{pdflscape}
\usepackage{tabu}
\usepackage{threeparttable}
\usepackage{threeparttablex}
\usepackage[normalem]{ulem}
\usepackage{makecell}
\usepackage{xcolor}
\ifLuaTeX
  \usepackage{selnolig}  % disable illegal ligatures
\fi
\usepackage{bookmark}
\IfFileExists{xurl.sty}{\usepackage{xurl}}{} % add URL line breaks if available
\urlstyle{same}
\hypersetup{
  pdftitle={Rmarkdown - PCA example},
  pdfauthor={R - course},
  hidelinks,
  pdfcreator={LaTeX via pandoc}}

\title{Rmarkdown - PCA example}
\author{R - course}
\date{18 November, 2024}

\begin{document}
\maketitle

\paragraph{Dependencies}\label{dependencies}

Pipeline depends on the following packages: `ggbiplot', `factoextra',
`kableExtra', `broom' and `ggrepel'. NOTE - in order to run Rmarkdown
each package needs to be already installed in you R for pipeline to
work. If you need to install required packages you can always use
install.pkg() function from `ANOVA Lecture R course.Rmd' script or by
using base function install.packages(). Each time you knit new clean
session is created thus, all packages for the pipeline are supposed to
be specified in the Rmarkdown otherwise knitting process will break.

\subsubsection{1. Data exploration}\label{data-exploration}

This document includes pipeline for data exploration with principal
component analysis or simply PCA. The basic principle of PCA is to
reduce dimensionality of the input data with many variables while
preserving max variation. By doing so original variables are transformed
to new set of variables called principal components. It is worth to
remember that number of resulting PCs is always less or equal to the
number of original variables. The first PC retains the max variation
from input data. In this example we'll use data MTCAR which is pre-load
in the base R.

\begin{itemize}
\item
  \textbf{MTCARS DATA SET} \emph{provides information extracted from the
  Motor Trend US magazine (1974), and comprises of fuel consumption and
  10 different aspects of automobile design and performance for 32
  automobiles (1973--74 models).}
\item
  \textbf{NOTE} that you can access all pre-loaded data set in R by
  simply calling \textbf{data()} in the command line.
\end{itemize}

\begin{Shaded}
\begin{Highlighting}[]
\CommentTok{\# Assign data to an object and show the table }
\NormalTok{df }\OtherTok{\textless{}{-}}\NormalTok{ mtcars}
\FunctionTok{kable}\NormalTok{(df[}\DecValTok{1}\SpecialCharTok{:}\DecValTok{5}\NormalTok{, ], }\AttributeTok{caption =} \StringTok{\textquotesingle{}MTCARS data set\textquotesingle{}}\NormalTok{, }\AttributeTok{format =} \StringTok{\textquotesingle{}latex\textquotesingle{}}\NormalTok{, }\AttributeTok{booktabs =}\NormalTok{ T) }\SpecialCharTok{\%\textgreater{}\%}
            \FunctionTok{kable\_styling}\NormalTok{(}\AttributeTok{latex\_options =} \FunctionTok{c}\NormalTok{(}\StringTok{\textquotesingle{}striped\textquotesingle{}}\NormalTok{,}\StringTok{"HOLD\_position"}\NormalTok{))}
\end{Highlighting}
\end{Shaded}

\begin{table}[H]
\centering
\caption{\label{tab:unnamed-chunk-1}MTCARS data set}
\centering
\begin{tabular}[t]{lrrrrrrrrrrr}
\toprule
  & mpg & cyl & disp & hp & drat & wt & qsec & vs & am & gear & carb\\
\midrule
\cellcolor{gray!10}{Mazda RX4} & \cellcolor{gray!10}{21.0} & \cellcolor{gray!10}{6} & \cellcolor{gray!10}{160} & \cellcolor{gray!10}{110} & \cellcolor{gray!10}{3.90} & \cellcolor{gray!10}{2.620} & \cellcolor{gray!10}{16.46} & \cellcolor{gray!10}{0} & \cellcolor{gray!10}{1} & \cellcolor{gray!10}{4} & \cellcolor{gray!10}{4}\\
Mazda RX4 Wag & 21.0 & 6 & 160 & 110 & 3.90 & 2.875 & 17.02 & 0 & 1 & 4 & 4\\
\cellcolor{gray!10}{Datsun 710} & \cellcolor{gray!10}{22.8} & \cellcolor{gray!10}{4} & \cellcolor{gray!10}{108} & \cellcolor{gray!10}{93} & \cellcolor{gray!10}{3.85} & \cellcolor{gray!10}{2.320} & \cellcolor{gray!10}{18.61} & \cellcolor{gray!10}{1} & \cellcolor{gray!10}{1} & \cellcolor{gray!10}{4} & \cellcolor{gray!10}{1}\\
Hornet 4 Drive & 21.4 & 6 & 258 & 110 & 3.08 & 3.215 & 19.44 & 1 & 0 & 3 & 1\\
\cellcolor{gray!10}{Hornet Sportabout} & \cellcolor{gray!10}{18.7} & \cellcolor{gray!10}{8} & \cellcolor{gray!10}{360} & \cellcolor{gray!10}{175} & \cellcolor{gray!10}{3.15} & \cellcolor{gray!10}{3.440} & \cellcolor{gray!10}{17.02} & \cellcolor{gray!10}{0} & \cellcolor{gray!10}{0} & \cellcolor{gray!10}{3} & \cellcolor{gray!10}{2}\\
\bottomrule
\end{tabular}
\end{table}

\textbf{Table 1. Description:} \textbf{mpg} -miles per gallon,
\textbf{cyl} -number of cylinders, \textbf{disp} -displacement,
\textbf{hp} -horse power, \textbf{draft} - rear axle ratio \textbf{wt} -
weight, \textbf{qsec} - quarter mile time, \textbf{vs} - tipe of the
engine block, \textbf{am} - transmission, \textbf{gear} - numer of gears
and \textbf{carb} - carburetors.

\pagebreak

\subsubsection{2. PCA analysis}\label{pca-analysis}

\begin{Shaded}
\begin{Highlighting}[]
\DocumentationTok{\#\# PCA work the best with numerical data }
\DocumentationTok{\#\# We\textquotesingle{}ll exclude categorical variables}
\NormalTok{df.pca }\OtherTok{\textless{}{-}} \FunctionTok{prcomp}\NormalTok{(df[,}\FunctionTok{c}\NormalTok{(}\DecValTok{1}\SpecialCharTok{:}\DecValTok{7}\NormalTok{,}\DecValTok{10}\NormalTok{,}\DecValTok{11}\NormalTok{)], }\AttributeTok{center =}\NormalTok{ T, }\AttributeTok{scale. =}\NormalTok{ T)}

\DocumentationTok{\#\# Plot PCA summary}
\FunctionTok{fviz\_eig}\NormalTok{(df.pca, }\AttributeTok{main =} \StringTok{\textquotesingle{}\textquotesingle{}}\NormalTok{, }\AttributeTok{ylab =} \StringTok{\textquotesingle{}\% of explained proportion\textquotesingle{}}\NormalTok{)}
\end{Highlighting}
\end{Shaded}

\begin{center}\includegraphics{PCA_analysis_files/figure-latex/unnamed-chunk-2-1} \end{center}

\textbf{\emph{PCA plot just shows the proportion of variance explained
by each principal component. Created PCA object (df.pca) contains
several information that you can access using \$ sign (exp:
df.pca\$center): center - mean, scale - sd, rotation - correlation
between initial variable and PC, x - values of each sample in terms of
PC.}}

\pagebreak

\subsubsection{3a. PCA plot}\label{a.-pca-plot}

\begin{Shaded}
\begin{Highlighting}[]
\CommentTok{\# ggbiplot {-} how variables relate to each other?}
\FunctionTok{ggbiplot}\NormalTok{(df.pca) }\SpecialCharTok{+} \FunctionTok{xlim}\NormalTok{(}\FunctionTok{c}\NormalTok{(}\SpecialCharTok{{-}}\DecValTok{2}\NormalTok{,}\DecValTok{2}\NormalTok{)) }\SpecialCharTok{+} \FunctionTok{ylab}\NormalTok{ (}\StringTok{\textquotesingle{}PC2 (23.12\%)\textquotesingle{}}\NormalTok{) }\SpecialCharTok{+} \FunctionTok{xlab}\NormalTok{(}\StringTok{\textquotesingle{}PC1 (62.84\%)\textquotesingle{}}\NormalTok{)}
\end{Highlighting}
\end{Shaded}

\begin{center}\includegraphics{PCA_analysis_files/figure-latex/unnamed-chunk-3-1} \end{center}

\textbf{\emph{More closely variables appear to each other more
correlated they are. Furthermore, variables carb, hp, cyl, disp and wt
are positively correlated with PC1 while the rest of the variables
present negative correlation towards PC1. To make this plot more
informative we should relate each point to corresponding car.}}

\pagebreak

\subsubsection{3b. Plot pca}\label{b.-plot-pca}

\begin{Shaded}
\begin{Highlighting}[]
\CommentTok{\# ggbiplot {-} how cars relate to each other?}
\FunctionTok{ggbiplot}\NormalTok{(df.pca) }\SpecialCharTok{+} \FunctionTok{xlim}\NormalTok{(}\FunctionTok{c}\NormalTok{(}\SpecialCharTok{{-}}\DecValTok{2}\NormalTok{,}\DecValTok{2}\NormalTok{)) }\SpecialCharTok{+}  \FunctionTok{ylab}\NormalTok{ (}\StringTok{\textquotesingle{}PC2 (23.12\%)\textquotesingle{}}\NormalTok{) }\SpecialCharTok{+} 
                  \FunctionTok{xlab}\NormalTok{(}\StringTok{\textquotesingle{}PC1 (62.84\%)\textquotesingle{}}\NormalTok{) }\SpecialCharTok{+} \FunctionTok{geom\_text\_repel}\NormalTok{(}\AttributeTok{label =} \FunctionTok{rownames}\NormalTok{(df))}
\end{Highlighting}
\end{Shaded}

\begin{center}\includegraphics{PCA_analysis_files/figure-latex/unnamed-chunk-4-1} \end{center}

\textbf{\emph{Here we can see which car is more similar to each other.
For example in top right corner we see cluster of 3 cars: Maserati Bora,
Ford Pantera and Ferrari Dino which makes sense as they are all sports
cars.}}

\pagebreak

\subsubsection{3c. Plot pca}\label{c.-plot-pca}

\begin{Shaded}
\begin{Highlighting}[]
\CommentTok{\# WE\textquotesingle{}ll manualy assign country of origin}
\NormalTok{df}\SpecialCharTok{$}\NormalTok{country }\OtherTok{\textless{}{-}} \FunctionTok{c}\NormalTok{(}\FunctionTok{rep}\NormalTok{(}\StringTok{"Japan"}\NormalTok{, }\DecValTok{3}\NormalTok{), }\FunctionTok{rep}\NormalTok{(}\StringTok{"US"}\NormalTok{, }\DecValTok{4}\NormalTok{), }\FunctionTok{rep}\NormalTok{(}\StringTok{"Europe"}\NormalTok{, }\DecValTok{7}\NormalTok{), }\FunctionTok{rep}\NormalTok{(}\StringTok{"US"}\NormalTok{,}
    \DecValTok{3}\NormalTok{), }\StringTok{"Europe"}\NormalTok{, }\FunctionTok{rep}\NormalTok{(}\StringTok{"Japan"}\NormalTok{, }\DecValTok{3}\NormalTok{), }\FunctionTok{rep}\NormalTok{(}\StringTok{"US"}\NormalTok{, }\DecValTok{4}\NormalTok{), }\FunctionTok{rep}\NormalTok{(}\StringTok{"Europe"}\NormalTok{, }\DecValTok{3}\NormalTok{),}
    \StringTok{"US"}\NormalTok{, }\FunctionTok{rep}\NormalTok{(}\StringTok{"Europe"}\NormalTok{, }\DecValTok{3}\NormalTok{))}

\CommentTok{\# ggbiplot {-} group cars based on the country of origin}
\FunctionTok{ggbiplot}\NormalTok{(df.pca, }\AttributeTok{ellipse =} \ConstantTok{TRUE}\NormalTok{, }\AttributeTok{ellipse.prob =} \FloatTok{0.68}\NormalTok{, }\AttributeTok{groups =}\NormalTok{ df}\SpecialCharTok{$}\NormalTok{country) }\SpecialCharTok{+}
    \FunctionTok{xlim}\NormalTok{(}\FunctionTok{c}\NormalTok{(}\SpecialCharTok{{-}}\DecValTok{2}\NormalTok{, }\DecValTok{2}\NormalTok{)) }\SpecialCharTok{+} \FunctionTok{ylab}\NormalTok{(}\StringTok{"PC2 (23.12\%)"}\NormalTok{) }\SpecialCharTok{+} \FunctionTok{xlab}\NormalTok{(}\StringTok{"PC1 (62.84\%)"}\NormalTok{) }\SpecialCharTok{+}
    \FunctionTok{geom\_text\_repel}\NormalTok{(}\AttributeTok{label =} \FunctionTok{rownames}\NormalTok{(df), }\AttributeTok{max.overlaps =} \ConstantTok{Inf}\NormalTok{)}
\end{Highlighting}
\end{Shaded}

\begin{center}\includegraphics{PCA_analysis_files/figure-latex/unnamed-chunk-5-1} \end{center}

\textbf{\emph{From here we can see that US and Japanese cars form 2
distinct clusters, while European cars are less tightly cluster however,
more similar to Japanese cars. Japanese and European cars are obviously
more fuel efficient, while US cars have more horse power, higher
displacement and more cylinders.}}

\end{document}
